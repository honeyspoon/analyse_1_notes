\documentclass{article}

\usepackage{amsfonts} 
\usepackage{amsmath}
\usepackage{amssymb}

\usepackage[utf8]{inputenc}

\newcommand{\notimplies}{\;\not\!\!\!\implies}
\newcommand{\notle}{\;\not\!\le}

\newcommand{\reals}{\mathbb{R}}
\newcommand{\naturals}{\mathbb{N}}
\newcommand{\rationals}{\mathbb{Q}}

\title{analyse 1}
\author{abderahmane bouziane}
\date{2022}

\begin{document}

\maketitle

6 septembre 2022

\section{Chapitre 1: Nombre réels $\reals$}

\subsection{Addition, multiplication et ordre}
on définit l'ensemble des nbrs reels noté $\reals$ de façon axiomatique.
On commence par qlqs axiomes auxquelles on en ajoute d’autres
L'ensemble R est muni de 2 operations

\begin{itemize}
    \item l'addition $(x, y) \rightarrow x + y$
    \item la multiplication $(x, y) \rightarrow x \cdot y$
\end{itemize}

Addition
\begin{itemize}
    \item A0) Fermeture: $\forall x, y \in \reals, x + y \in \reals$
    \item A1) Commutativité: $\forall x, y \in \reals, x + y = y + x$
    \item A2) Associativité: $\forall x, y, z \in \reals, x + (y + z) = (y + x) + Z$
    \item A3) Element neutre: $\forall x \in \reals, \exists! 0. x + 0 = x$
    \item A4) Inverse additif : $\forall \in \reals \exists -x, x + (-x) = 0$
\end{itemize}

Multiplication
\begin{itemize}
    \item M0) Fermeture: $\forall x, y \in \reals, x \cdot y \in \reals$
    \item M1) Commutativité: $\forall x, y \in \reals, x \cdot y = y \cdot x$
    \item M2) Associativité: $\forall x, y, z \in \reals, x \cdot (y \cdot z) = (y \cdot x) \cdot z$
    \item M3) Element neutre: $\forall x \in \reals, \exists! 1. x + 1 = x$
    \item M4) Inverse multiplicatif : $\forall \in \reals \exists x^{-1}, x \cdot x^{-1} = 1$
\end{itemize}

Commun
\begin{itemize}
    \item D) Distributivité: $\forall x, y, z \in \reals, x \cdot (y + z) = x \cdot y + x \cdot z$
\end{itemize}

Remarques
\begin{itemize}
    \item L’élément neutre est unique
    \item L'inverse n'est pas spécifie unique, mais c'est deductible
    \item de $A0, A2, A3$ on peut montrer que l'inverse additif est unique
    \item de $M0, M2, M3$ on peut montrer que l'inverse multiplicatif est unique

\end{itemize}

Ex: MQ (montrez que) l'inv. add.  est unique.\\
on suppose que c'est faux\\
ie $\exists x \in \reals \; \text{pour lequel} \exists (-x) et (-\hat{x}) \in \reals$ \\
$x + (-x) = 0,  x + (-\hat{x} = 0)$ \\
but: prouver que $-x$ et $-\hat{x}$ sont égaux \\

\begin{equation}
\begin{aligned}
    -x &= (-x + 0) \; \text{par A3} \\
       &= (-x + (x + - \hat{x})) \; \text{par H (hypothèse)} \\
       &= ((-x + x) + - \hat{x}) \; \text{par A2} \\
       &= (0 + - \hat{x}) \; \text{par H} \\
    -x &= - \hat{x} \; \text{par A3} \\
\end{aligned}
\end{equation}

Ex: MQ $x \cdot 0  = 0 \; \forall x \in \reals$ \\

\begin{equation}
\begin{aligned}
    0 &=  (x \cdot 0) + (- (x \cdot 0)) \; \text{par A4} \\
      &=  (x \cdot (0 + 0)) + (- (x \cdot 0)) \; \text{par A3} \\
      &=  (x \cdot 0 + x \cdot 0) + (- (x \cdot 0)) \; \text{par D} \\
      &=  (x \cdot 0) + (x \cdot 0 + - (x \cdot 0)) \; \text{par A2} \\
      &=  (x \cdot 0) + 0 \; \text{par A4} \\
    0 &=  (x \cdot 0) \; \text{par A3} \\
\end{aligned}
\end{equation}

Exercice: MQ $-1 \cdot x = -x$ \\

\begin{equation}
\begin{aligned}
    x \cdot 0 &= 0 \; \text{par Ex précédant} \\
    x \cdot (1 + -1) &= 0 \; \text{par A4} \\
    x \cdot 1 + x \cdot -1 &= 0 \; \text{par D} \\
    x + x \cdot -1 &= 0 \; \text{par M4} \\
    x + -x + x \cdot -1 &= 0 + -x \; \text{par ?} \\
    (x + -x) + x \cdot -1 &= -x \; \text{par A2 et A3} \\
    x \cdot -1 &= -x \; \text{par A4} \\
\end{aligned}
\end{equation}

notation: $x - y = x + (-y)$\\
Question: A0 - A4, M0 M4, D sont=ils suffisants pour définir les nombre reels?\\
De A3 et M3 on déduit que $\{0, 1\} \in \reals$\\
De A0 $\{0, 1, 1 + 1, 1 + 1 + 1 \ldots\} = \{0, 1, 2, 3 \ldots\} = \{0\} \cap \naturals \subset \reals$ \\
De A4 $\{\ldots, -3, -2, -1, 0, 1, 2, 3 \ldots\} = \mathbb{Z} \subset \reals$ \\ 
De M4 $\{\ldots, \frac{-3}{2}, \ldots\} = \{ \frac{m}{n} | m \in \mathbb{Z} , n \in \naturals \} = \rationals \subset \reals$ \\ 

Donc $\rationals$  et $\reals$ vérifient tous les deux A0 - A4, M0 M4, D \\
$\rationals \subset \reals$\\
$\rationals \ne \reals$ ?\\

NB: les complexes complexes aussi les axiomes \\

Pour les distinguer il faut des axiomes supplémentaires \\

Remarque  $\{ \frac{m}{n} \;|\; m \in \mathbb{Z} , n \in \naturals \}$ \\
le signe  est au numérateur\\
on ne divise jamais par 0\\

8 septembre\\
cours 2\\

def: Une relation d'ordre sur un ensemble $E$ est une relation q'on note ici $\le$ vérifient les propriétés suivantes\\
i) réflexivité\\
$$ x \le x \; \forall x \in E$$ \\
ii) antisymétrique \\
$$ x \le y \; \text{et} \; y \le x \; \text{alors} \; x = y$$ \\
iii) transitivité\\
$$ x \le y \; \text{et} \; y \le z \; \text{alors} \; x \le z$$ \\

Axiomes d'ordre sur $\reals$ \\

O) Ordre: $\reals$ est muni d'une relation d'ordre note $\le$, vérifiant les propriétés suivantes \\
O1) $0 \le 1$ et $0 \ne 1$\\
O2) $\forall x \in \reals$ (au mois) une des affirmations suivantes est vraie\\
$0 \le x$ ou $x \le 0$ \\
O3) $x \le y, x + z \le y + z \forall x,y,z \in \reals$ \\
O4) $0 \le x, 0 \le y, 0 \le x \cdot y$ \\

Remarque:\\
$\mathbb{C}$ ne vérifie pas (O), Donc, $\reals \ne \mathbb{C}$ \\
$\rationals$ vérifie (O), car $\rationals \subset \reals$\\
On ne sait toujours pas si $\rationals \ne \reals$\\

Notation:\\
on écrit $x < y \rightarrow x \le y, \; x \ne y$\\
on écrit $x \ge y \rightarrow y \le x$\\
on écrit $x > y \rightarrow y < x$\\

MQ: $0 < x \Longleftrightarrow - x < 0$ \\
split en 2 \\
$0 < x \Longrightarrow - x < 0$ et $-x < 0 \Longrightarrow 0 < x$ \\

1) $0 < x \Longrightarrow - x < 0$ \\
Hypothèse (H1): $0 < x$

\begin{equation}
\begin{aligned}
    -x &=  -x\\
    -x &=  0 + -x &\;         & \text{par A3} \\
    -x &=  0 + -x &\le \; 0 + -x + x\; &\text{par O3} \\
    -x &=  0 + -x \le \; 0 \;  &     & \text{par H1} \\
    -x &\le \; 0 \;  &     & \text{} \\
\end{aligned}
\end{equation}

par contradiction \\
Hypothèse (H2): $ -x = 0 $ \\
\begin{equation}
\begin{aligned}
    x &= \; x + 0 \; & \text{par A3} \\
    &= x + -x \; &\text{par H2} \\
    &= 0 \; &\text{par A4} \\
    \text{contradiction avec } H1
\end{aligned}
\end{equation}\\

donc Hypothèse (H2) est fausse, $-x \ne 0$ \\
donc $-x < 0$ \\

% TODO:
Exercice: \\
2) $-x < 0 \Longrightarrow 0 < x$ \\
Hypothèse: $-x < 0$ \\

\begin{equation}
\begin{aligned}
    ...\\\
    x &\ge 0 \\
\end{aligned}
\end{equation}\\
Ex: MQ $0 < x \implies 0 < x^{-1}$\\
par contradiction\\
Hypothèse H3:\\
$\exists x \in \reals, x < 0, \; x^{-1} = 0 $\\
\begin{equation}
\begin{aligned}
    1 &= x \cdot x^{-1} \\
    1 &= x \cdot 0 \text{ par M3} \\
    1 &= 0 \text{ par exemple précédant} \\
    &\text{contradiction} \\
    &\text{donc H3 est faux}, \\
    &x^{-1} = 0 \\
\end{aligned}
\end{equation}\\

deuxième partie:\\
par contradiction\\
Hypothèse H4: $x^{-1} > 0$
\begin{equation}
\begin{aligned}
    x^{-1} &< 0\\
    0 &< - x^{-1} \text{ par exemple précédant} \\
    0 &< - x^{-1} \cdot x \text{ par O4} \\
    0 &< - 1 \text{ par exemple précédant et M4} \\
    &\text{contradiction} \\
    &\text{donc H4 est faux}, \\
    &0 \le x^{1}\\
    &0 < x^{-1} \text{ en combinant résultat précédant}\\
\end{aligned}
\end{equation}\\

% TODO:
Exercice: MQ \\
$$y^{-1} < x^{-1} \implies 0 < x < y$$

Definition:\\
soit $x \in \reals$, la valeur absolue de x est noté $|x|$\\

$$
|x| = 
\begin{cases}
    x ,& x > 0 \\
    -x ,& x < 0
\end{cases}
$$
proposition\\
$\forall x \in \reals, |x| \ge 0$ \\
$ |x| = 0 \Longleftrightarrow x = 0$
demonstration triviale
% TODO:

proposition\\
$\forall x \in \reals, -|x| \le x \le |x|$ \\
\begin{equation}
\begin{aligned}
    \text{si } x = 0\\
    - |0| = 0 = |0| \\
    \\
    \text{si } x > 0\\
    |x| = x > 0 \\
    |x| = x > 0 > -x \text{ par exemple}\\
    - |x| = x < |x| \text{ par exemple}\\
    \\
    \text{si } x < 0\\
    -x > 0 > x \text{ par exemple}\\
    % TODO:
    ????\\\
\end{aligned}
\end{equation}\\

Thm inégalité du triangle \\
$$|x + y| \le |x| +|y| \; \forall x, y \in \reals$$
donc par la proposition précédant \\
$-|x| \le x \le |x|$\\
$-|y| \le y \le |y|$\\
en additionnant les 2 on obtiens: \\
$-(|x| + |y|) \le x + y \le |x| + |y|$ \\

si $x + y > 0$ \\
$x + y = |x+y| \le |x| + |y|$\\
$|x + y| \le |x| + |y|$ \\

si $x + y < 0,$\\
??? \\\

Attention\\
si $x \le y \notimplies |x + z| \le |y + z| $ \\
contre-exemple \\
$ x = -2, y = 1 z = -1 $ \\
$ x < y $ \\
$ | - 2 - 1 | = 3  \notle | 1 - 1 | = 0$ 

\subsection{Induction mathématique}

principe du bon ordre\\
tout ensemble non vide $ E \subset \naturals$ possède un plus petit element\\

proposition: soit $E \subset \naturals$ tel que \\
i) $1 \in E$\\
ii) $n \in E \implies n+1 \in E$\\

alors $E  = \naturals$

par contradiction\\
Hypothèse H5\\
$\naturals \setminus E \ne \emptyset$\\
cet ensemble possède eun plus petit element, par le principe du bon ordre\\
$\exists n_0 \in \naturals \setminus E, n_0 \text{ est le plus petit element}$ \\

Puisque $1 \in E$\\
$n_0 \ne 1, n_0 > 1$\\
$n_0 - 1 \in \naturals, n_0 - 1 \in E$ \\
par ii) $n_0 -1 + 1 \in E$\\

contradiction car $n_0 \in \naturals \setminus E$
H5 est faux\\
$\naturals \setminus E = \emptyset$

Thm: principe d'induction\\
$$\forall n \in \naturals, (P(n))$$\\
i) $(P(1))$ est vraie\\
ii) $(P(n)) \implies P(n + 1), \forall n \in \naturals$ \\
alors $(P(n)) \text{ est vraie } \forall n \in \naturals$

demonstration découle directement de la proposition précédente en considerant
$$E = \{n \in \naturals \; | \; (P(n)) \text{ est vraie}\}$$\\

Ex: Inégalité de Bernoulli \\
\begin{equation}
\begin{aligned}
(1 + x)^n \le 1 + nx \; \forall n \in \naturals\\
1) (P(1)): (1 + x)^1 \ge 1 + x \text{ vraie} \\
2) \text{en assumant vraie } (P(n)) \\
\end{aligned}
\end{equation}\\

\begin{equation}
\begin{aligned}
(1 + x)^{n + 1} &= (1 + x)^n (1 + x) \\
&\ge (1 + nx)(1 + x)\\
&\ge 1 + (n + 1)x + n x ^2 \text{ car } nx ^2 \ge 0\\
\end{aligned}
\end{equation}\\
3) par le principe de l'induction l’inégalité de Bernoulli est vraie\\
 
Remarque: \\
on utilise $x \le -1$, pour preserver le sens de l’inégalité.\\

Ex: MQ: $2^3 + 4^3 + 6^3 + ... + (2n)^3 = 2n^2(n+1)^2 \; \forall n \in \naturals$\\
3) par le principe d'induction l'egalite est vrai $\forall n \in \naturals$\\
\begin{equation}
\begin{aligned}
1) (P(1)):  2^3 = 2(2)^2\text{ vraie} \\
2) \text{en assumant vraie } (P(n)) \\
\end{aligned}
\end{equation}\\

\begin{equation}
    \begin{aligned}
        ... + (2(n + 1))^3 &= 2n^2(n+1)^2 + (2(n + 1))^2 \cdot (2n+ 1) \\
        &= 2(n + 1)^2(n^2 + 4(n+ 4))\\
        &= 2(n + 1)^2(n + 2)^2\\
        &= 2(n + 1)^2(n + 1 + 1)^2\\
    \end{aligned}
\end{equation}

Exercice: pour $x \in ]-1, 1[$ MQ $\Sigma_{k=0}^n x^k = \frac{1-x^{n+1}}{}$

\subsection{Suprémum et infimum}

\end{document}